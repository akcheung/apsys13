\usepackage{amsmath}
\usepackage{times-lite,color}
\usepackage{paralist}
\usepackage{proof}
\usepackage{semantic}
\usepackage{graphicx}
\usepackage{xspace}
\usepackage{caption}
\DeclareCaptionType{copyrightbox} % hack to make caption happy with sigplanconf
\usepackage{subcaption}
%\usepackage[caption=false]{subfig}
\usepackage{xcolor}
\usepackage{amssymb}
\usepackage{pifont}
\usepackage{listings}
\usepackage{amsthm}
\usepackage[T1]{fontenc}
\usepackage{minibox}
%\newcommand*{\subfigureautorefname}{\figureautorefname}
\usepackage{sfmath}
\IfFileExists{.usepdfcomments}
{\usepackage[icon=Comment,color=notecolor,hoffset=-5pt,voffset=8pt]{pdfcomment}
\def\usepdfcomment{}}

\usepackage[sfcode,small]{ttquot}

\def\thepage{\arabic{page}} % delete for final camera-ready

%\usepackage{inconsolata}

\renewcommand*\ttdefault{txtt}


%%%%%% Typeset code %%%%%%
\definecolor{javapurple}{rgb}{0.5,0,0.35}
\definecolor{javagreen}{rgb}{0,0.6,0}

\lstset{columns=fullflexible}
\lstset{language=java, basicstyle=\fontfamily{lmss}\selectfont\small, 
        keywordstyle=\color{javapurple}\fontfamily{lmss}\selectfont\bfseries,
        commentstyle=\tt\slshape,
        }

%\lstset{language=java, basicstyle=\ttfamily, 
%        keywordstyle=\color{javapurple}\tt\bfseries},
%         commentstyle=\tt\slshape}

%\lstset{deletekeywords={int,float,double,char,if,else,return,class,void,this,
%                        new,for,public}}

\lstset{numbers=left, firstnumber=1, numberstyle=\tiny\color{gray}}
\lstset{alsoletter={:}, morekeywords={:APP:,:DB:}, escapeinside={@}{@}}
\lstset{showstringspaces=false}


%%%%%% AST %%%%%%
%\newcommand{\bnf}{\ensuremath{\ | \ }}
\renewcommand{\sp}{\ | \ }
\newcommand{\logicAnd}{\wedge}
\newcommand{\logicOr}{\vee} 
\newcommand{\code}[1]{\lstinline[language=java]{#1}}

%%%%%% Comments %%%%%%
\newcommand{\textred}[1]{\textcolor{red}{#1}}
\ifx\noeditingmarks\undefined%
\ifx\usepdfcomment\undefined%
\newcommand{\pgwrapper}[4]{%
\def\argone{#1}%
\def\none{None}%
\ifx\argone\none%
\textred{#2: #3}%
\else%
\textred{#1 [#2: #3]}%
\fi}%
\else%
\definecolor{notecolor}{rgb}{1.0,1.0,0.7}%
\definecolor{srmcolor}{rgb}{1.0,1.0,0.7}%
\definecolor{aslcolor}{rgb}{1.0,0.8,0.7}%
\definecolor{accolor}{rgb}{0.7,1.0,1.0}%
\newcommand\pgwrapper[4]{%
\def\argone{#1}%
\def\none{None}%
\ifx\argone\none%
\pdfcomment[author=#2,color=#4]{#3}%
\else%
\pdfmarkupcomment[author=#2,color=#4]{#1}{}%
\pdfmargincomment[author=#2]{#3}\fi}%
\fi
\else
\newcommand{\pgwrapper}[4]{#1}
\fi

\newcommand{\wu}[2][None]{\pgwrapper{#1}{WU}{#2}{srmcolor}}
\newcommand{\lr}[2][None]{\pgwrapper{#1}{LR}{#2}{aslcolor}}
\newcommand{\ac}[2][None]{\pgwrapper{#1}{AC}{#2}{accolor}}

\IfFileExists{.nocomments}
    {\usepackage[disabled]{aplcomments}}
    {\IfFileExists{.inlinecomments}
        {\usepackage[inline]{aplcomments}}
        {\usepackage{aplcomments}}}


\newcommenter{wu}{1.0,1.0,0.3}
\newcommenter{ac}{0.4,1.0,1.0}
\newcommenter{lr}{0.4,0.4,1.0}


%%%%%%% Labels %%%%%%
\newcommand{\figlabel}[1]{\label{f:#1}}
\newcommand{\seclabel}[1]{\label{s:#1}}
\newcommand{\secref}[1]{Sec.~\ref{s:#1}}  % for use in text
\newcommand{\Secref}[1]{Sec.~\ref{s:#1}}  % for start of sentence
\newcommand{\figref}[1]{Fig.~\ref{f:#1}}     % for use in text
\newcommand{\Figref}[1]{Figure~\ref{f:#1}}   % for start of sentence
\newcommand{\Figrefs}[2]{Figures~\ref{f:#1} and~\ref{fig:#2}}
\newcommand{\figrefsMany}[1]{Figs.~\ref{f:#1}}
\newcommand{\lstref}[1]{line~\ref{lst:#1}}
\newcommand{\thmlabel}[1]{\label{t:#1}}
\newcommand{\thmref}[1]{Thm.~\ref{t:#1}}
\newcommand{\applabel}[1]{\label{ap:#1}}
\newcommand{\appref}[1]{Appendix~\ref{ap:#1}}


%%%%%% Bibliography %%%%%%
\usepackage{hyperref}
  \let\oldthebibliography=\thebibliography
  \let\endoldthebibliography=\endthebibliography
  \renewenvironment{thebibliography}[1]{%
    \begin{oldthebibliography}{#1}%
      \setlength{\parskip}{0ex}%
      \setlength{\itemsep}{0ex}%
  }%
  {%
    \end{oldthebibliography}%
  }


%%%%%% Misc %%%%%%
\newcommand{\C}[1]{\lstinline!#1!}
\providecommand{\abs}[1]{\lvert#1\rvert}

%\renewcommand{\b}[1]{\lstinline{#1}}
\renewcommand{\b}[1]{\ensuremath{{\sf{#1}}}}
%\renewcommand{\t}[1]{\text{#1}}
%\renewcommand{\b}[1]{\t{\lstinline{#1}}}

% relational algebra
\newcommand{\select}[2]{\ensuremath{\sigma_{#1}(#2)}}
\newcommand{\selectSym}{\b{\sigma}}

\newcommand{\project}[2]{\ensuremath{\pi_{#1}(#2)}}
\newcommand{\projectSym}{\b{\pi}}

\newcommand{\join}[2]{\ensuremath{\bowtie_{#1}\hspace{-0.02in}(#2)}}
\newcommand{\joinP}[2]{\ensuremath{\bowtie'_{#1}\hspace{-0.02in}(#2)}}
\newcommand{\joinSym}{\b{\bowtie}}

\newcommand{\append}{\b{append}}
\newcommand{\sort}[2]{\ensuremath{\b{sort}_{#1}(#2)}}
\newcommand{\sortSym}{\b{sort}}

\newcommand{\unique}{\b{unique}}

\renewcommand{\top}[2]{\ensuremath{\b{top}_{#1}(#2)}}
\newcommand{\topSym}{\b{top}}

\renewcommand{\min}{\b{min}}
\renewcommand{\max}{\b{max}}
\renewcommand{\sum}{\b{sum}}
\newcommand{\agg}{\b{agg}}
\newcommand{\size}{\b{size}}
%\newcommand{\True}{\b{{\bf True}}}
\newcommand{\True}{\b{True}}
\newcommand{\False}{\b{False}}
\newcommand{\Query}{\b{Query}}

\newcommand{\get}[2]{\ensuremath{\b{get}_{#1}(#2)}}
\newcommand{\getSym}{\b{get}}
\newcommand{\getHelper}{\b{getHelper}}

\newcommand{\contains}{\b{contains}}

\newcommand{\concat}{\b{cat}}

\newcommand{\tail}{\b{tail}}
\newcommand{\head}{\b{head}}

\newcommand{\AND}{\wedge}
\newcommand{\OR}{\vee}
\newcommand{\NOT}{\neg}

% input language
\newcommand{\ite}[3]{\b{if}(#1)~\b{then}~#2~\b{else}~#3}
\newcommand{\while}[2]{\b{while}(#1)~\b{do}~#2}
\newcommand{\SKIP}{\b{skip}}
\newcommand{\assert}[1]{\b{assert}~#1}

% SQL commands
\newcommand{\SELECT}{~\b{SELECT}~}
\newcommand{\SELECTDISTINCT}{~\b{SELECT~DISTINCT}~}
\newcommand{\FROM}{~\b{FROM}~}
\newcommand{\WHERE}{~\b{WHERE}~}
\newcommand{\UNION}{~\b{UNION~ALL}~}
\newcommand{\IN}{~\b{IN}~}
\newcommand{\ORDER}{~\b{ORDER~BY}~}
\newcommand{\AS}{~\b{AS}~}
\newcommand{\getOrder}{\b{Order}}

\newcommand{\qbs}{{\scshape Qbs}\xspace}
\newcommand{\sembrack}[1]{[\![#1]\!}
\newcommand{\transFn}{\b{Trans}}
\newcommand{\id}{\b{id}}


\newcommand{\VC}{\b{VC}}



%\usepackage{float}
%\floatstyle{boxed}
%\restylefloat{figure}

\newtheorem{thm}{Theorem}
\newtheorem{definition}{Definition}
% theorems and definitions w/o numbering
\newtheorem*{theorem*}{Theorem}
\newtheorem*{definition*}{Definition}

\newcommand{\etal}{\textit{et al.}\@\xspace}
\newcommand{\eg}{\textit{e.g.}\@\xspace}
\newcommand{\ie}{\textit{i.e.}\@\xspace}
\newcommand{\Sk}{\textsc{Sketch}}
\newcommand{\cegis}{\textsc{cegis}}

\newcommand{\cmark}{\ding{51}}

\newcommand{\arrow}{\rightarrow}
